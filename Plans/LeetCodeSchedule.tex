\documentclass{article}
\usepackage{enumitem}
\usepackage{datetime}
\begin{document}

\section*{LeetCode Problem Categories by Technique}

\subsection*{1. Array Manipulation}
\textbf{Techniques:} Sliding window, two pointers, prefix sums, hashing.
\begin{itemize}[noitemsep]
    \item Two Sum
    \item Maximum Subarray
    \item Product of Array Except Self
    \item Subarray Sum Equals K
    \item Merge Intervals
\end{itemize}

\subsection*{2. Strings}
\textbf{Techniques:} String manipulation, pattern matching, two pointers, hashing.
\begin{itemize}[noitemsep]
    \item Longest Substring Without Repeating Characters
    \item Valid Anagram
    \item Group Anagrams
    \item Longest Palindromic Substring
    \item String to Integer (atoi)
\end{itemize}

\subsection*{3. Linked List}
\textbf{Techniques:} Slow and fast pointers, dummy nodes, reversing a list.
\begin{itemize}[noitemsep]
    \item Reverse Linked List
    \item Merge Two Sorted Lists
    \item Linked List Cycle
    \item Add Two Numbers
    \item Intersection of Two Linked Lists
\end{itemize}

\subsection*{4. Stack and Queue}
\textbf{Techniques:} Monotonic stack, breadth-first search (BFS), depth-first search (DFS).
\begin{itemize}[noitemsep]
    \item Valid Parentheses
    \item Daily Temperatures
    \item Evaluate Reverse Polish Notation
    \item Sliding Window Maximum
    \item Implement Stack Using Queues
\end{itemize}

\subsection*{5. Binary Tree}
\textbf{Techniques:} DFS, BFS, recursion, iterative traversal.
\begin{itemize}[noitemsep]
    \item Binary Tree Inorder Traversal
    \item Maximum Depth of Binary Tree
    \item Path Sum
    \item Lowest Common Ancestor of a Binary Tree
    \item Serialize and Deserialize Binary Tree
\end{itemize}

\subsection*{6. Binary Search Tree}
\textbf{Techniques:} BST properties, recursion, in-order traversal.
\begin{itemize}[noitemsep]
    \item Validate Binary Search Tree
    \item Insert into a Binary Search Tree
    \item Kth Smallest Element in a BST
    \item Lowest Common Ancestor of a BST
    \item Binary Search Tree Iterator
\end{itemize}

\subsection*{7. Recursion and Backtracking}
\textbf{Techniques:} Generate permutations/combinations, search space pruning.
\begin{itemize}[noitemsep]
    \item Subsets
    \item Permutations
    \item Combination Sum
    \item Sudoku Solver
    \item Word Search
\end{itemize}

\subsection*{8. Dynamic Programming}
\textbf{Techniques:} State definition, transitions, memoization, tabulation.
\begin{itemize}[noitemsep]
    \item Climbing Stairs
    \item House Robber
    \item Longest Increasing Subsequence
    \item Coin Change
    \item Longest Common Subsequence
\end{itemize}

\subsection*{9. Greedy}
\textbf{Techniques:} Local optimization leads to global solution.
\begin{itemize}[noitemsep]
    \item Jump Game
    \item Gas Station
    \item Meeting Rooms II
    \item Partition Labels
    \item Minimum Number of Arrows to Burst Balloons
\end{itemize}

\subsection*{10. Binary Search}
\textbf{Techniques:} Divide and conquer, searching in sorted arrays.
\begin{itemize}[noitemsep]
    \item Binary Search
    \item Search in Rotated Sorted Array
    \item Find Minimum in Rotated Sorted Array
    \item Koko Eating Bananas
    \item Median of Two Sorted Arrays
\end{itemize}

\subsection*{11. Graphs}
\textbf{Techniques:} BFS, DFS, union-find, shortest path algorithms.
\begin{itemize}[noitemsep]
    \item Number of Islands
    \item Clone Graph
    \item Course Schedule
    \item Minimum Spanning Tree
    \item Dijkstra's Algorithm
\end{itemize}

\subsection*{12. Bit Manipulation}
\textbf{Techniques:} Bitwise operations, XOR, masking.
\begin{itemize}[noitemsep]
    \item Single Number
    \item Hamming Distance
    \item Counting Bits
    \item Reverse Bits
    \item Subsets (Bitmasking)
\end{itemize}

\subsection*{13. Sliding Window}
\textbf{Techniques:} Dynamic window resizing, maintaining invariants.
\begin{itemize}[noitemsep]
    \item Minimum Window Substring
    \item Longest Substring with At Most Two Distinct Characters
    \item Sliding Window Maximum
    \item Subarray Product Less Than K
    \item Find All Anagrams in a String
\end{itemize}

\subsection*{14. Two Pointers}
\textbf{Techniques:} Moving two pointers towards each other or independently.
\begin{itemize}[noitemsep]
    \item Two Sum II (Input Array Is Sorted)
    \item Container With Most Water
    \item Three Sum
    \item Remove Duplicates from Sorted Array
    \item Trapping Rain Water
\end{itemize}

\subsection*{15. Sorting}
\textbf{Techniques:} Merge sort, quicksort, custom comparators.
\begin{itemize}[noitemsep]
    \item Merge Intervals
    \item Sort Colors
    \item Kth Largest Element in an Array
    \item Meeting Rooms
    \item Largest Number
\end{itemize}

\subsection*{16. Heap (Priority Queue)}
\textbf{Techniques:} Min-heap, max-heap, custom comparators.
\begin{itemize}[noitemsep]
    \item Kth Largest Element in a Stream
    \item Top K Frequent Elements
    \item Find Median from Data Stream
    \item Merge K Sorted Lists
    \item Meeting Rooms II
\end{itemize}

\subsection*{17. Math and Number Theory}
\textbf{Techniques:} Modular arithmetic, combinatorics, prime numbers.
\begin{itemize}[noitemsep]
    \item Fibonacci Number
    \item Greatest Common Divisor of Strings
    \item Sqrt(x)
    \item Pow(x, n)
    \item Count Primes
\end{itemize}

\subsection*{18. Union-Find (Disjoint Set)}
\textbf{Techniques:} Path compression, union by rank.
\begin{itemize}[noitemsep]
    \item Number of Connected Components in an Undirected Graph
    \item Redundant Connection
    \item Graph Valid Tree
    \item Accounts Merge
    \item Smallest String With Swaps
\end{itemize}

\subsection*{19. Trie (Prefix Tree)}
\textbf{Techniques:} Word prefix storage and retrieval.
\begin{itemize}[noitemsep]
    \item Implement Trie (Prefix Tree)
    \item Add and Search Word
    \item Word Search II
    \item Replace Words
    \item Maximum XOR of Two Numbers in an Array
\end{itemize}

\subsection*{20. Interval Problems}
\textbf{Techniques:} Sorting intervals, merging intervals.
\begin{itemize}[noitemsep]
    \item Merge Intervals
    \item Insert Interval
    \item Non-Overlapping Intervals
    \item Meeting Rooms
    \item Minimum Number of Arrows to Burst Balloons
\end{itemize}

\subsection*{21. Divide and Conquer}
\textbf{Techniques:} Recursive problem decomposition.
\begin{itemize}[noitemsep]
    \item Merge Sort
    \item Quick Sort
    \item Search in a Rotated Sorted Array
    \item Median of Two Sorted Arrays
    \item Maximum Subarray (Divide and Conquer Version)
\end{itemize}

\section*{Study Plan Calendar}

\textbf{Study Plan:} Starting from \textbf{14/12/2024} to \textbf{21/4/2025}. The study plan is organized into continuous intervals, with one-week revisions after completing a topic.

\begin{enumerate}[noitemsep]
    \item \textbf{Week 1 (14/12/2024 - 20/12/2024):} Array Manipulation
    \item \textbf{Week 2 (21/12/2024 - 27/12/2024):} Strings
    \item \textbf{Week 3 (28/12/2024 - 03/01/2025):} Linked List
    \item \textbf{Week 4 (04/01/2025 - 10/01/2025):} Stack and Queue
    \item \textbf{Week 5 (11/01/2025 - 17/01/2025):} Binary Tree
    \item \textbf{Week 6 (18/01/2025 - 24/01/2025):} Binary Search Tree
    \item \textbf{Week 7 (25/01/2025 - 31/01/2025):} Recursion and Backtracking
    \item \textbf{Week 8 (01/02/2025 - 07/02/2025):} Dynamic Programming
    \item \textbf{Week 9 (08/02/2025 - 14/02/2025):} Greedy
    \item \textbf{Week 10 (15/02/2025 - 21/02/2025):} Binary Search
    \item \textbf{Week 11 (22/02/2025 - 28/02/2025):} Graphs
    \item \textbf{Week 12 (01/03/2025 - 07/03/2025):} Bit Manipulation
    \item \textbf{Week 13 (08/03/2025 - 14/03/2025):} Sliding Window
    \item \textbf{Week 14 (15/03/2025 - 21/03/2025):} Two Pointers
    \item \textbf{Week 15 (22/03/2025 - 28/03/2025):} Sorting
    \item \textbf{Week 16 (29/03/2025 - 04/04/2025):} Heap (Priority Queue)
    \item \textbf{Week 17 (05/04/2025 - 11/04/2025):} Math and Number Theory
    \item \textbf{Week 18 (12/04/2025 - 18/04/2025):} Union-Find (Disjoint Set)
    \item \textbf{Week 19 (19/04/2025 - 21/04/2025):} Trie (Prefix Tree)
    \item Revisions will take place weekly after completing each topic.
\end{enumerate}

\end{document}
